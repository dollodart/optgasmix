\documentclass{article}
\usepackage{amsmath}
\usepackage{gensymb}
\usepackage{listings}
\usepackage{siunitx}
\usepackage{hyperref}
\lstset{language=Python}
\usepackage{biblatex}
\addbibresource{ref.bib}

\usepackage{geometry}
\geometry{margin=1in}

\title{Determination of the Optimal Gas Mixture for Heat Transfer}
\author{David Ollodart}

\begin{document}
%
\maketitle
\tableofcontents
%
\section{Problem Statement}

Gases are poor conductors of heat relative to liquids. However, they
can be used to directly cool a product in many cases where a liquid
heat transfer needs a shell. Some applications in semiconductor
processing require direct cooling of devices by gases, and though
limited by reactivity considerations, there are still many eligible
inert gases. These applications are at standard pressure or less in
vacuum systems and near standard temperature or greater, well within the
ideal gas range for any inert gas. It is relative simple to evaluate the
transport properties of gases at low density where the kinetic theory
of gases is valid. Also, in this particular case, the flow is usually
forced-convection, unconfined, and laminar, which is further ideal,
since transport calculations for turbulent flow are harder to make and
more uncertain.

\section{Non-thermodynamic and Transport Considerations}

These considerations are either qualitative (hard constraint) or factor
into an optimization as linear terms.

\subsection{Chemical Reactivity}

There are really only nitrogen and the noble gases as valid heat
transfer gases which are inert. In operations where hydrogen is anyways
used, it is possible to also use hydrogen, since the reduction of the
substrate is already allowed. The other gases have deficiencies: water
vapor (possible oxidant, and condenses at room temperature), halides
(highly reactive), oxygen (highly oxidating), hydrocarbons (cause some
contamination).

\subsection{Price (due to separations cost)}

The dry composition of air is mostly nitrogen (around 77\%), then
oxygen (around 22\%), then significant amount of Argon (around 0.9\%),
then trace gases. It is therefore easy (and inexpensive) to extract
and purify nitrogen, oxygen, and argon from the atmosphere through
fractional distillation. Even Krypton is around 1 ppm and can be
fractionally distilled from air. Other gases are components of common
chemicals. Most notably hydrogen can be easily generated even from
water, but also all of the halides, sulfides, and oxides may be derived
as byproducts of industrially important separations processes. 

Only helium is a gas which is not appreciably in the atmosphere and
not a component of any chemical, and so has to be non-renewably mined
from deposits where it is trapped inside shale and other geological
features (though in the solar system, it is the second most abundant
chemical). Do note, though, that as the lowest molecular weight noble
gas it has some extreme properties that may make it desirable.

\section{Molecular Theories for the Properties of Gases}

The relevant factors can be discussed based on the molecular theories by
which the equations are derived.

There are the following factors, sorted in approximately decreasing
order of importance:

\begin{enumerate}
\item Atomic mass(es)
\item Number of Atoms
\item Chemical valence
\item Structure
\end{enumerate}

The governing equation for momentum transport, the steady-state
Navier-Stokes equation, neglecting changes in pressure, has two terms:

\begin{equation}
\rho \mathbf{v} \cdot \nabla \mathbf{v} = \mu \nabla \cdot \nabla \mathbf{v} \,.
\end{equation}

Here the molecular model is discussed which determine $\mu$,
the coefficient on the viscous transport term. The steady-state heat
transfer equation is, neglecting viscous dissipation,

\begin{equation}
\mathbf{v} \cdot \nabla (C_p(T) T) = k \nabla \cdot \nabla T \,.
\end{equation} % C_p is molar specific heat capacity

Here the molecular model is discussed which determines $k$, the
coefficient on the heat diffusion term.

Thermodynamics factors into transport properties because the same
molecular modes that allow for heat to be absorbed allow for heat to
be transferred, so the heat capacity appears in thermal conductivity
properties. Also, obviously the density as a thermodynamic property of
the gas appears as an important term.

Later in \S\ref{sec:transport-eqns} the macroscopic, convective
transport for the other terms are discussed.

\subsection{Thermodynamics}

When discussing specific properties, like the specific heat capacity,
it is crucial to establish whether it is specific to mass, volume,
or number (the latter two being proportional at any temperature for a
gas). Also, which thermodynamic variable is being held constant, since
$C_p = C_v + R$. Here, heat capacity always refers to number specific
rate of change of enthalpy with respect to temperature at constant
pressure, unless otherwise qualified. Note number and volume specific
trends are the same and so used interchangeably.

The volume specific heat capacity increases with the size of the
molecule, especially because of vibrational and rotational modes. Atomic
mass has an affect on vibrational modes and their energies, too. However
the volume specific heat capacity should be independent of the atomic
mass, and that is realized in noble gases which all have a heat capacity
of very nearly $5R/2$. Of course, the mass specific heat capacity therefore
decreases by the proportion of the mass increase.

\subsection{Microscopic (Viscous/Diffusive) Transport}

\subsubsection{Energetic Parameters and Size}

The energetic parameter of interaction is much lower for noble gases
than others, as expected by chemical valence (as an example, the
ratio of the energetic parameter of helium to that of hydrogen is
approximately $3:19$). The collision cross-section increases with the
energetic parameter, as expected, since two molecules are more likely
to be "captured" in an interaction if they interact more strongly
(here likely means, at all the same and some farther distances of
approach). All transport properties decreases with increasing energetic
parameter and collision cross-section, since the

The collision cross-section increases with the size and therefore mass
(when mass is by adding atoms, rather than changing atomic mass),
because bigger molecules are more likely to hit each other. These
decrease the transport properties for both heat and momentum. However,
bigger molecules also have higher heat capacities and more modes
by which to transfer energy for every collision, so the thermal
conductivity happens to increase with molecule size.

% In fact, as one can see in $k$, there is a proportionality to heat
% capacity which increases with molecular size, and therefore molecular
% mass.

\subsubsection{(Atomic) Mass}

The mass of the molecule is a factor in momentum transport at a
macroscopic level, in the convective heat transport and boundary layer
analysis. As the quantitative section shows, a higher mass improves
heat transfer macroscopically, though the net effect of increasing mass
can be to decrease transport due to microscopic or molecular factors on
transport properties. This section discusses mass changes due to changes
in the atomic mass, since there are other effects from increasing the
mass by increasing the number of atoms and therefore size.

Since the Maxwell--Boltzmann distribution specifies a distribution of
kinetic energies, the velocity of the particles decreases with the root
of atomic mass. So even atomic mass changes in the same period tend
to change transport properties, though there are other factors (the
homologous series of noble gases gives 10.2, 35.7, 122.4, 170, and
234 for the Lennard-Jones parameter, and 2.576, 2.789, 3.432, 3.679,
4.009 for the effective radius. The Lennard-Jones parameter changes
significantly because of London dispersion interactions). From BSL the
transport properties are more exactly

\begin{align}
\mu &= \frac{5}{16} \frac{\sqrt{\pi k_B m T}}{\pi \sigma^2 \Omega} \tag{BSL 1.4-14}\,\\
k &= (C_p + \frac 54  k_B )  \frac{\mu}{m} \tag{BSL 9.3-15} \label{eqn:kpoly}\,.
\end{align}

This shows the property of molecular momentum transport, viscosity,
increases with the square root of mass. This is because if the
kinetic energy distribution is fixed, then $mv^2/2$ and equivalently
$p^2/2m$ is fixed (fixing kinetic energy is equivalent to fixing
temperature). Therefore the average momentum of a particle increases
by the square root of mass when the mass is increased at a given
temperature.

However, the property of molecular heat transfer, the thermal
conductivity, decreases with the square root of mass. This is because
the velocity decreases with the square root of mass. The mean free path
is independent of atomic mass, so when the velocity decreases with the
square root of mass, the frequency of collisions between molecules
decreases by the same. More exactly, there is the following for
intermolecular collision rate:

\begin{equation}
\gamma = \frac{P}{k_B T} \sigma^2 \sqrt{\frac{8 k_B T}{\pi m}} \,.
\end{equation}

%The decrease is due to a decrease in the frequency of collisions while
%how far molecules get between collisions remains the same.

For heat transfer, considering only the collision of gas molecules at
microscopic scale, a small, low-atomic weight, and inert molecule is
best for transport. For this reason, helium gas is often used. However,
because of both convective flow and coupled heat-and-mass transport for
diffusive flow which increase with increasing atomic mass, such a gas
may not be optimum.

\section{Mixing Rules}

The mixing rule for transport properties is non-linear (BSL Transport
Phenomena, 2nd Ed).  

\begin{align}
\mu_\text{mix} &= \sum_{\alpha=1}^N \frac{x_\alpha \mu_\alpha}{\sum_\beta x_\beta \Phi_{\alpha\beta}}\,, \tag{BSL 1.4-15}\\
\Phi_{\alpha\beta} &= \frac{1}{\sqrt{8}} \left(1 + \frac{M_\alpha}{M_\beta}\right)^{-1/2}
\left[1 + \left(\frac{\mu_\alpha}{\mu_\beta}\right)^{1/2}\left(\frac{M_\beta}{M_\alpha}\right)^{1/4}\right]^2 \,.\tag*{}
\end{align}

As BSL notes in section 1.4 and asks the reader to work out in problem
1A.2 (though the nearest problem is actually 1A.4), the mixture property
can be more extreme than the properties of the single components. In the
example they give data showing the viscosity of a mixture of 75\%
hydrogen and 25\% dichlorodifluoromethane as 135.1 micropoise, compared
to 124 micropoise for dichlorodifluoromethane and 88.4 micropoise for
hydrogen.

For this reason, optimization of transport properties should include
composition as an independent variable. However, given the significant
difference between hydrogen and dichlorodifluoromethane's molar masses
(the ratio is around 60), it isn't necessarily representative of most
gas mixtures. When the components are similar, often the non-linear
equation still gives an interpolation of the mixture property from
the component properties. One can observe that when the ratios of
viscosities and molar massses are both unity, the $\Phi_{\alpha\beta}$
value weighting the composition $x_\beta$ becomes just unity, and the
mixture rule becomes equal to a weighted arithmetic mean.

If you optimize with prices, however, then even if the mixture property
interpolates the component properties the optimum may be a mixture.

\section{Transport Equations}\label{sec:transport-eqns}

In general correlations for the Nusselt number may be power relations
such as $N_{Nu} = N_{Re}^a N_{Pr}^b$. There is (\cite[eqn.~11.4-2]{deen1998analysis})

\begin{equation}
N_{Nu} = \begin{cases}
\sqrt{N_{Re}} & N_{Re} \gg 1 \\
\sqrt{N_{Re}N_{Pr}} & \text{fluid-fluid} \,,\\
(N_{Re} N_{Pr})^{1/3} & N_{Re} < 1, N_{Pr} \gg 1, \text{fluid-solid}\,,\\
N_{Re}^{1/2} N_{Pr}^{1/3} & N_{Re} \gg 1, N_{Pr} \gg 1, \text{fluid-solid}\,,\\
N_{Re}^{1/2} N_{Pr}^{1/2} & N_{Re} \gg 1, N_{Pr} \ll 1, \text{fluid-solid}\,.
\end{cases} \label{eqn:Nu-corrs}
\end{equation}

%if Re > 100: # Re >> 1
%    return Re**0.5
%elif phase == 'fluid':
%    return Re**0.5 * Pr**0.5
%elif Re < 1 and Pr > 100 and phase == 'solid': # Re <<1 or ~ `, Pr >> 1
%    return Re**0.33 * Pr**0.33
%elif Re > 100 and Pr > 100 and phase == 'solid':
%    return Re**0.5 * Pr**0.33
%elif Re > 100 and Pr < 0.01 and phase == 'solid':
%    return Re**0.5 * Pr**0.5

Here

\begin{align}
N_{Re} &= \frac{\rho v D}{\mu} \label{eqn:Re}\,,\\
N_{Pr} &= \frac{\mu C_p}{k m} \label{eqn:Pr}\,.
\end{align}

From (\ref{eqn:Nu-corrs}), (\ref{eqn:Re}), and (\ref{eqn:Pr}), one can
find from the definition of $N_{Nu} = hL/k$ an effective heat transfer
coefficient ratio generally as follows (the heat capacity is number specific):


\begin{equation}\label{eqn:main-equation}
\frac{h_A}{h_B} = \left(\frac{k_A}{k_B}\right)^{1-b} 
\left(\frac{\mu_A}{\mu_B}\right)^{b-a} 
\left(\frac{C_{p, A}}{C_{p, B}}\right)^b 
\left(\frac{m_A}{m_B}\right)^{a-b} \,.
\end{equation}

For ideal gases, the Prandtl number is independent of the pressure, because the
thermodynamic properties of viscosity, heat capacity, and thermal conductivity
only depend on one variable in the equation of state, from molecular theory
originating as temperature (I say this rather than simply that "they are a
function of temperature", since one could substitute in a pressure-volume
dependence by the equation of state). However, pressure factors in
significantly to the momentum transfer in terms of the density, to which it is
proportional to pressure. The above equation is for comparing two gases (or gas
mixtures) under the same conditions, in which case, $\rho_A/\rho_B = m_A/m_B$.
One must use an arithmetic mean for the masses, as the thermodynamic properties
of a gas mixture are arithmetic means.

For air the Prandtl number is around 0.7, which is quite near 1. The
correlations given are limited to transport regimes with limiting values of
some dimensionelss groups, as the inequalities show. However, one always
expects some power law scaling, so the question is really of the relative
magnitudes of $b$ and $a$, which determine, for two of these quotients, whether
the ratio is greater than or less than $1$.  For all reported correlations ist
is shown that $a \geq b$, and the incompressible flow over a flat plate follows
this as well, with $a=\frac 12$ and $b=\frac 15$ (see
\S\ref{ref:incompressible-flat-plate}).

\subsection{Further Simplifications and Comparison of Noble Gases}\label{sec:compare-noble}

For gases, $k \propto C_p \mu$, up to an additive constant, so that one can
(approximately) simplify the above to

$$\frac{h_A}{h_B} = \left(\frac{\mu_A}{\mu_B}\right)^{1-a} \left(\frac{C_{p, A}}{C_{p, B}}\right)^b \left(\frac{m_A}{m_B}\right)^{a-b} \,.$$

One can further put in mass dependencies for the viscosity, which go as the
square root (though atomic diameter collision cross-section factors are here
relevant, too). 

$$\frac{h_A}{h_B} = \left(\frac{\sigma_A^2 \Omega_A}{\sigma_B^2 \Omega_B}\right)^{a-1} \left(\frac{C_{p, A}}{C_{p, B}}\right)^b \left(\frac{m_A}{m_B}\right)^{1/2 + a/2 -b} \,.$$

For monatomic gases the ratio of number specific heat capacities is 1. This
applies in comparing noble gases. In what remains, one can see that there is a
dependence for which the mass ratios require $a > 2b - 1$, which is realized in
almost any case since $b \leq 0.5$. However, since $a$ is very often less than
one, and since as monatomic gas mass increases so do collision cross-section
and diameter increase (the former because of dispersion interactions), the net
effect for heat transfer is not obvious. This is tested (see
\texttt{test-noble-gases} in the \texttt{test} directory), and it is obtained
that, relative to helium as the maximum, neon has 56\% the heat transfer
coefficient, argon 35\%, and krypton 26\%. Evidently the mass dependence is
outdone by the radius and collision cross-section dependence.

\subsection{Boundary Layer Convective Effects}

It is required to explain the correlations (at a qualitative level) for the heat transfer coefficients
which were simply taken from reference. From \cite[eqn.~7.19]{incropera1996fundamentals} the boundary
layer thickness is inversely proportional to the root of Reynold's number,

\begin{equation}\label{eqn:boundary-layer-thickness}
\delta(x) = \frac{5 x}{\sqrt{N_{Re}(x)}} \,.
\end{equation}

So the boundary layer thickness is inversely proportional to the
root of density from (\ref{eqn:Re}). For this reason, though higher
atomic weight gases are worse for molecular level, or diffusive,
transport, they are better for convective transport. In fact, just
as the boundary layer decreases with the root of density, so do gas
particle velocities decrease with the root of density, though this won't
precisely cancel out because the terms don't appear in a ratio for the
transport properties.

Consequently, in convection dominated transport, which is the most common
industrial flow case, higher atomic weight gases may be better, though this
doesn't bear out in one case as already found for noble gases at least when
laminar (see \S \ref{sec:compare-noble}). Do note, though, that Reynolds number
dependence increases from 1/2 to 4/5 from laminar to turbulent flow (compare
equations (7.30) and (7.36) \cite{incropera1996fundamentals}).

\section{Modeling Flat Plate Heat Transfer}

The above discussions are excellent for understanding the transport
coefficients which have geometry- and flow-independent scaling, scaling only
with thermodynamic variables of temperature and pressure and molecular
properties (when comparing different gases and gas mixtures). Here some
analytical results are treated for incompressible flows, and extensions to
numerical computation of compressible flows and non-ideal geometries is given.

\subsection{Radiative Heat Transfer for Wafer}

It should be noted that the purpose of cooling with gases would generally to bring
the wafer temperature down to room temperature from slight elevations. It isn't
needed for very high temperatures, since the wafer should cool down by
radiative heat transfer. In fact, to approximate the relative effect of
convective and radiative heat transfer, there is the ratio

$$ \frac{\sigma (T^4 - T_\infty^4)}{h(T - T_\infty)} \,,$$

where $\sigma$ is the Stefan-Boltzmann constant and $h$ is a representative
heat transfer coefficient. Using a representative heat transfer coefficient of
$\SI{10}{W/(m^2\cdot K}$, this ratio is unity already at $\SI{400}{K}$. Emissivity factor
corrections may decrease the readiative heat transfer significantly, especially
when the mean wavelength changes by Wien's displacement law so that even a
constant spectral emissivity function would result in a change in effective
emissivity. If the effective emissivity is $0.5$ then the ratio is unity at $\SI{560}{K}$.

\subsection{Solution of the Isothermal Flat Plate (Incompressible Fluid)}\label{ref:incompressible-flat-plate}

The isothermal flat plate problem for incompressible flow has the Blasius
solution, which is an analytical solution in the laminar regime, which applies
when cooling with gases under many regions of interest.  Note this flat plate
flow is parallel to the plate surface. The local Nusselt number, which is
position dependent, is $N_{Nu}(x) = C N_{Re}(x)^{1/2} N_{Pr}^{1/3}$ where $C
\approx 1/3$ (\cite[eqn.~7.30]{incropera1996fundamentals}). This requires only
that $N_{Pr} \geq 0.6$, which is satisfied by most gases. The local Nusselt
number can be averaged over a given distance $L$. In particular,
$\overline{N_{Nu}} = \frac 29 N_{Re}^{1/2} N_{Pr}^{1/3}$. This is anyways
irrelevant since the ratios, of interest for determining a gas property, are
independent of such factors. One has for the flat plate

\begin{equation} 
\frac{h_A}{h_B} = \left(\frac{k_A}{k_B}\right)^{2/3} 
\left(\frac{\mu_A}{\mu_B}\right)^{-1/6} 
\left(\frac{C_{p, A}}{C_{p, B}}\right)^{1/3} 
\left(\frac{m_A}{m_B}\right)^{1/6} \,.
\end{equation} % check powers

The average Nusselt number is good to have, rather than
just the ratio, in order to check the validity of the isothermal or lumped
capacitance assumption. In particular, $N_{Bi} = hL/k_{solid}$, and there
results from substitution of the Nusselt number definition $N_{Bi} =
k/k_{solid} N_{Nu}$. The time constant in a lumped capaciatance model is found as 

$$\tau = \frac{h}{L C_p \rho/\mathcal{M}} \,,$$

where as is convention throughout this document $C_p$ is number-specific.

\subsubsection{Validity of Isothermal Assumption in Wafers}\label{sec:validity-isothermal}

Wafers in modern fabrication are $\SI{300}{mm}$ in diameter and around
$\SI{1}{mm}$ in thickness. This aspect ratio is so extreme that it is rarely
observed in any other application.

In the direction perpendicular to the wafer the lumped capacitance assumption
is certainly valid due to this extreme aspect ratio. What may be less valid is
thermal homogeneity in the direction parallel to the wafer. Silicon has a
thermal conductivity similar to metals, around $\SI{148}{W/(m\cdot K)}$ at
$\SI{300}{K}$, though it decreases with increasing temperature to as little as
$\SI{30}{W/(m\cdot K)}$ at $\SI{1000}{K}$ (while more phonon modes are
accessible at higher temperatures, at higher temperatures it can be they
interfere with each other). Using the correlations for one gas mixture (see 
\texttt{test-flat-plate.py}), I find $N_{Bi} = 0.04$, and it is only required that
$N_{Bi} < 0.1$ for the lumped capacitance (isothermal) assumption to be valid. 

Under the same conditions the time constant for the lumped capacitance model is
around 3 milliseconds. For a wafer initially at $\SI{700}{\degree C}$, cooling
to $\SI{25}{\degree C}$ when the ambient temperature is $\SI{20}{\degree C}$
takes for such a time constant only $\SI{15}{ms}$. There is then a question for
the motivation of this work as applied to semiconductor wafers, since any gas
composition would be sufficient for cooling purposes, if cooling were ever
needed. Still, there remain other interesting applications of the general
question when the time constant is not necessarily so small, and very often the
surrounding equipment, which has much higher thermal mass, may need to be
cooled between processing steps.

\section{Uniformly Random Sampling of Composition Space}

In general, there is a simple but inefficient algorithm to generate
a mesh of arbitrary dimension $m$ with fineness $n$ in $\Theta(n^m)$
time. It is to iterate through every $m$ permutation of the elements
of the range $0, 1/n, \ldots, 1$, that is, where $X_m$ is the
$m$-dimensional Cartesian product, to iterate through every sequence in
$X_m (0, 1/n, \ldots, 1)$. Then, check if the permutation sums to 1, and
accept it if it does, otherwise rejecting it.

This inefficiency to generate sample points is actualy not irrelevant
since it is less expensive to do calculations on the coordinates
$n^{m-1}$, provided they are $O(1)$ which for these algebraic equations
they are, than it is to generate them in $O(n^m)$. While asymptotic
growth doesn't take into account constant factors which are very
significant, still there is the mathematically interesting question of
how to uniformly randomly sample $m$ variables provided they sum to $1$
and hence there are $m-1$ degrees of freedom. 

Ternary plots allow you to uniformly sample by the triangular grid
generated from intersections of 60\degree lines drawn from the
equilateral triangle axes which are interior to the equilateral triangle.
In particular, from each axis you draw lines parallel to the preceding
axis, assigning a rotational order like clockwise. The intersections,
tracing back to the axis linearly spaced, yield a composition tuple
whose sum is 1. This turns out to be equivalent to the following simple
iterative procedure:

\begin{lstlisting}
linspace = tuple(x/10 for x in range(11))
for x1 in linspace:
    for x2 in linspace:
        if 1 - x1 - x2 > 0:
            combs.append(x1, x2, 1 - x1 - x2)
\end{lstlisting}

However, an analogous routine does not work for quaternary
composition. More generally, it may be expected that barycentric
coordinates provide a means to uniformly randomly sample subject to the
constraint.

\subsection{Method by Sampling from Flat Dirichlet Distribution}

I have not found a means using geometry and barycentric coordinates to randomly
sample the quaternary phase space. One can randomly sample by using the
Dirichlet distribution with all unity concentration coefficients, which is
called the flat Dirichlet distribution. One randomly samples from an exponential
distribution (which is an instance of the gamma distribution for general
concentration coefficients) with probability density $e^{-x}$. Then one take
thats and transform for every random sample $x_i$, where one has sampled $n$
times to generate one random sample for an $n$-composition, into the following:

\begin{equation}
y_i = \frac{x_i}{\sum_{j=1}^n x_j}
\end{equation}

While not grid sampling, uniform random sampling achieves the same aim of
uniformity. The length scale over which random sampling is approximately
uniform, by which it is meant there is guaranteed to be at least one sample
point, depends on the number of random samples drawn. The length scale must
scale as $1/n^{1/3}$ and must be greater than the grid that would be formed by
the same number of samples. Theoretically an exact quantification is
interesing, that is, determine the distribution of $\max(\delta_{ij})$ where
$\delta_{ij} = x_i - x_j$ and $x_0 = 0$, $x_{n+1} = 1$ for one dimension and
then higher dimensions when the $x_i$ are realizations of a uniform random
distribution on $[0, 1]$.  But the method of random sampling is so much faster
than grid sampling that it is irrelevant in this application.

Necessarily, the distribution of any one composition will not be
uniformly random when the number of composition variables is greater
than 2. The constraint (in statistics, the support of the distribution)
makes it so a uniform grid has some values less likely than others. In
particular, as you go toward $1$ for any given composition, the
number of other possible compositions becomes only $1$ because of the
constraint. Visually, this can be seen in the 3-composition case where
there is less area (phase space) near the triangle vertices than in the
triangle interior, and likewise for the vertices and interior of the
tetrahedron in the 4-composition case.

Generally, the symmetric Dirichlet distribution with some $\alpha$ can be used.
For the symmetric Dirichlet Distribution (of which the flat Dirichlet
distribution is the case $\alpha = 1$) there is

\begin{align}
\mu = \frac 1n \,,\\
\nu = \frac{n-1}{n+1} \,,
\end{align}
%where $n$ is the dimension of the space.

both of which are independent of $\alpha$. The mode is equal to the mean
for the symmetric Dirichlet distribution. Hence there is no evident
reason to use the general symmetric Dirichlet distribution over the flat
Dirichlet distribution.

\subsection{Other Methods}

Knuth (\cite[\S 3.4.1]{taocp}) gives a method to determine a random point on an
$n$-dimensional sphere with radius one, which is what is needed for a
composition space of $n$ components (as opposed to a volume of a simplex of
dimension $n-1$). It consists in using generalized polar coordinates to obtain
points randomly oriented about the origin in $n$-dimensional space. These
points can be projected along the radial line to the unit sphere to get a
solution. He cites \cite[302]{brownmodernmathematics1956} as the first
suggestion of this method.  This solution has the advantage of not being a
rejection sampling algorithm, and so not wasting any samples which are
generated.

In order to make a uniform grid in a hyperdimensional space on the surface of a
hyperdimensional sphere, one might, rather than uniformly vary the angles,
instead uniformly vary their arcsines or arccosines. In this way, one would
obtain uniform arcs on the surface, which by their intersections make a uniform
grid.

\section{Optimization Search}

Grid enumeration is the most inefficient way to determine an optimum, though it
is not expensive here for ternary gas compositions and there is scientific
interest in evaluating the entire composition space. Any optimization search
can be used, and the composition constraint can be imposed simply by
substituting the last composition variable as 1 minus the sum of all others.
Therefore secant or other non-derivate methods, first derivative methods like
Newton's method or the more general line search, or second derivative methods
can be used. Imposing constraints on non-linear problems can be difficult if
temperature, pressure, flow geometry, or flow velocity are to be optimized in
addition to composition. Constraints are generally required since the governing
equations are only valid for some ranges of the dimensionless groups. But
methods such as sequential quadratic programming can be used in this case.

\section{Conclusion}

The main result is given in eqn.~\ref{eqn:main-equation}. A figure of merit for
a gas can be defined in terms of its quantities appearing in that equation,
which is dimensioned, but allow comparison between more than one gas. This is
done in the \texttt{test} directory using the powers of the Blasius solution to
the flat plate problem (\texttt{test-flat-plate.py}). A small discussion was
made on uniform sampling of high dimensional composition space, with reference
to search methods, in the case that high dimensional (quaternary or higher)
optima may need to be found.

\printbibliography

\end{document}
